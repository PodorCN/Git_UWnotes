\documentclass{report}
\usepackage{xcolor,tikz}
\title{Math235 Notes}
\usepackage{listings}
\usepackage{xcolor,color}
\usepackage{courier}
\usepackage{inconsolata}
%\renewcommand*\familydefault{\ttdefault}
\lstset{numbers=left,
numberstyle=\tiny,
keywordstyle=\color{blue!70}, commentstyle=,
frame=shadowbox,
rulesepcolor=\color{red!20!green!20!blue!20},
breaklines=true,
extendedchars=true
}


\definecolor{mykeycode}{RGB}{138,74,11}
\definecolor{mybg}{RGB}{64,64,64}
\definecolor{mycm}{RGB}{170,215,208}
\definecolor{mycode}{RGB}{219,219,219}


\lstset{language=C++,
numbers=left,
                basicstyle=\color{mycode}\ttfamily,
                keywordstyle=\color{mykeycode}\ttfamily,
                stringstyle=\color{red}\ttfamily,
                commentstyle=\color{mycm}\ttfamily,
                morecomment=[l][\color{magenta}]{\#},
                backgroundcolor=\color{mybg}, 
                morekeywords={*,string.cin}
}
\author{mzx!}
\date{\today}
\newcommand{\ibx}[1]{\framebox[1.1\width]{ #1 }}
\usepackage{amsmath,amssymb}
\usepackage{hyperref}
\usepackage{enumitem} 
\hypersetup{
    colorlinks,
    citecolor=black,
    filecolor=black,
    linkcolor=black,
    urlcolor=black
}

\PassOptionsToPackage{svgnames}{xcolor}
\usepackage{tcolorbox}
\usepackage{lipsum,xcolor,color}
\tcbuselibrary{skins,breakable}
\usetikzlibrary{shadings,shadows}

\definecolor{title_color}{HTML}{ea7dc7}
\definecolor{back_color}{HTML}{f7e8e8}

\newcounter{defboxctr}
\newenvironment{defbox}{%
\refstepcounter{defboxctr}% increment the environment's counter
    \tcolorbox[beamer,%
    noparskip,breakable,
    colback=back_color,colframe=title_color,%
    title={Definition \thedefboxctr}]}%
    {\endtcolorbox}
    \numberwithin{defboxctr}{section}





\PassOptionsToPackage{svgnames}{xcolor}
\usepackage{tcolorbox}
\usepackage{lipsum,xcolor,color}
\tcbuselibrary{skins,breakable}
\usetikzlibrary{shadings,shadows}

\definecolor{title_color}{HTML}{ea7dc7}
\definecolor{back_color}{HTML}{f7e8e8}

\newcounter{defboxctr}
\newenvironment{defbox}{%
\refstepcounter{defboxctr}% increment the environment's counter
    \tcolorbox[beamer,%
    noparskip,breakable,
    colback=back_color,colframe=title_color,%
    title={Definition \thedefboxctr}]}%
    {\endtcolorbox}
    \numberwithin{defboxctr}{section}

\begin{document}

\maketitle
\tableofcontents
\chapter{Eigenvectors and Diagonalization}
\section{Similar Matrices}
\begin{quote}
Introducing Jennifer. She has her own language to represent any vector\\In her language, \{$\begin{bmatrix}1\\0\end{bmatrix}$,$\begin{bmatrix}0\\1\end{bmatrix}$\} equals \{$\begin{bmatrix}3\\2\end{bmatrix}$,$\begin{bmatrix}1\\-1\end{bmatrix}$\} in OUR Co-ordination System
\end{quote}

\begin{itemize}
\item Let $\mathbb B$ be the basis of \textbf{Jennifer's} Co-ordination System
\item Let L be the linear-transformation from \textbf{Jennifer's} Co-ordination System to \textbf{MY} Co-ordination System.
\item Let \textbf{Matrix A} be the standard Linear-Transformation of L $A=[L]$,in this case, $A=\begin{vmatrix}3&1\\2&-1\end{vmatrix}$
\item $[\vec x]_{\mathbb{B}}$ represent how \textbf{Jennifer} represent \ibx{$\vec x$ from \textbf{MY} Co-ordination System} using her language
\item Since we have P, \ibx{$\vec{x}$ in Jennifer's language} (i.e. $[\vec{x}]_{\mathbb{B}}$) multiple by P is $\vec{x}$ in Our System \ibx{$P[\vec{x}]_{\mathbb{B}}= \vec{x}$}
\item Inverse, $[\vec{x}]_\mathbb{B} = P^{-1}\vec{x}$,$(P^{-1})$ convert any vector in out language into Jennifer's language
\item
\end{itemize}
\chapter{Applications of Orthogonal Matrices}
\section{Orthogonal Similarity}
\subsection{Application}
\begin{itemize}
\item Find one of the real eigenvalues $C(\lambda )=\det(A-\lambda I)=0$
\item Find the corresponding eigenvector$(\vec v_1)$ $A-\lambda I \to{RREF}$
\item Extend $\vec v_1$ to orthonormal basis of $R^n$, Usually $I$, but not always
\item Calculate $P^T_1AP_1 = \begin{vmatrix}\lambda & \vec{b}^T\\\vec{0}&A_1\end{vmatrix}$
\item Inductively working on $A_1$,Notices that $\begin{vmatrix}\lambda & \vec{b}^T\\\vec{0}&A_1\end{vmatrix}$ is already upper triangular
\item We are looking for Orthogonal Matrix $Q$ such that $Q^TA_1Q = T_1$
\end{itemize}
\end{document}