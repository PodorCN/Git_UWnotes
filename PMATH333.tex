\documentclass{report}
\usepackage{xcolor,tikz}
\title{PMATH 333 Preview Notes}
\usepackage{listings}
\usepackage{xcolor,color}
\usepackage{courier}
\usepackage{inconsolata}
%\renewcommand*\familydefault{\ttdefault}
\lstset{numbers=left,
numberstyle=\tiny,
keywordstyle=\color{blue!70}, commentstyle=,
frame=shadowbox,
rulesepcolor=\color{red!20!green!20!blue!20},
breaklines=true,
extendedchars=true
}


\definecolor{mykeycode}{RGB}{138,74,11}
\definecolor{mybg}{RGB}{64,64,64}
\definecolor{mycm}{RGB}{170,215,208}
\definecolor{mycode}{RGB}{219,219,219}


\lstset{language=C++,
numbers=left,
                basicstyle=\color{mycode}\ttfamily,
                keywordstyle=\color{mykeycode}\ttfamily,
                stringstyle=\color{red}\ttfamily,
                commentstyle=\color{mycm}\ttfamily,
                morecomment=[l][\color{magenta}]{\#},
                backgroundcolor=\color{mybg}, 
                morekeywords={*,string.cin}
}
\author{mzx!}
\date{\today}
\newcommand{\ibx}[1]{\framebox[1.1\width]{ #1 }}
\usepackage{amsmath,amssymb}
\usepackage{hyperref}
\usepackage{enumitem} 
\hypersetup{
    colorlinks,
    citecolor=black,
    filecolor=black,
    linkcolor=black,
    urlcolor=black
}

\PassOptionsToPackage{svgnames}{xcolor}
\usepackage{tcolorbox}
\usepackage{lipsum,xcolor,color}
\tcbuselibrary{skins,breakable}
\usetikzlibrary{shadings,shadows}

\definecolor{title_color}{HTML}{ea7dc7}
\definecolor{back_color}{HTML}{f7e8e8}

\newcounter{defboxctr}
\newenvironment{defbox}{%
\refstepcounter{defboxctr}% increment the environment's counter
    \tcolorbox[beamer,%
    noparskip,breakable,
    colback=back_color,colframe=title_color,%
    title={Definition \thedefboxctr}]}%
    {\endtcolorbox}
    \numberwithin{defboxctr}{section}





\PassOptionsToPackage{svgnames}{xcolor}
\usepackage{tcolorbox}
\usepackage{lipsum,xcolor,color}
\tcbuselibrary{skins,breakable}
\usetikzlibrary{shadings,shadows}

\definecolor{title_color}{HTML}{ea7dc7}
\definecolor{back_color}{HTML}{f7e8e8}

\newcounter{defboxctr}
\newenvironment{defbox}{%
\refstepcounter{defboxctr}% increment the environment's counter
    \tcolorbox[beamer,%
    noparskip,breakable,
    colback=back_color,colframe=title_color,%
    title={Definition \thedefboxctr}]}%
    {\endtcolorbox}
    \numberwithin{defboxctr}{section}

\begin{document}

\maketitle
\tableofcontents
\chapter{The Real Number System}
**This Chapter Will Be Skipped Due To The Easiness of The Chapter**
\chapter{Sequences in R}
\section{Limits of Sequences}
\paragraph{Infinite Sequence}is a function whose domain is $N$. For example,\\\ibx{1,$1\over 2$,$1\over 4$... represents the sequence$\{{1\over {2^{n-1}}}\}_{n \in \mathbf N}$}\\\ibx{-1,1,-1,1...represents the sequence $\{{(-1)^n}\}_{n\in \mathbf N}$}\\
\paragraph{•} It is important not to confuse a \textbf{Sequence} with the \textbf{Set}. For example,\\\ibx{Sequence ${\{x_n\}_{n\in \mathbf N}}$ is 1,2,3,4....}\\\ibx{Set $\{{x_n:n\in \mathbf N\}}$ could be 2,1,3,4...} \\
\paragraph{•} Also, the \textbf{Sequence} 1,-1,1,-1, is infinite, but the \textbf{Set} $\{{(-1)^n : n\in\mathbf N}
\}$ has only 2 points.


\begin{defbox}
A sequence of real numbers {$x_n$} is said to \emph{converge} to a real number $a\in \mathbf R$ $\iff$ for every $\varepsilon > 0$,there is an $ N\in \mathbf N$ (which in general depends on $\varepsilon$),such that
$$ n\ge N \qquad \text{implies} \qquad |x_n-a| < \varepsilon$$
\end{defbox}
\paragraph{Interchangeable notation} We shall use the following phrases and notation interchangeably\\
a)$\{x_n\}$ converges to $a$\\
b)$x_n$ converges to $a$;\\
c)$a = \lim_{n\to\infty} x_n$;\\
d)$x_n\to a$ as $n\to \infty$\\
e)the \emph{limit}  of {$x_n$} exists and equals a\\ 
\subsection{Holds for large}
Let $\mathcal{P}_n$ be a property indexed by $\mathbf{N}$.We shall say that $\mathcal{P}_n$  \emph{Holds for large n} if there is an $N \in \mathbf{N}$ such that $\mathcal{P}_n$ is true for all $n\geq N$. \\
Hence a loose summary of Definition 2.1 is that $x_n$ converges to $a$ $\iff |x_n - a|$ is small for large $n$

\subsection{At Most One Limit}
A sequence can have at most one limits
\paragraph{Proof}Suppose that $\{x_n\}$ converges to both $a$ and $b$. By definition, given $\varepsilon > 0$ there is an integer $N$ such that $n\geq N$ implies $|x_n - a| < {\varepsilon \over 2} $and $|x_n - b| < {\varepsilon \over 2} $. Thus it follows
$${|a-b|} \leq {|a-x_n| + |x_n-b|}<\varepsilon $$
Since $|a-b| < \varepsilon$, we conclude that a = b\footnote{By Theorem 1.9}

\subsection{Subsequence}
\begin{defbox}
By a \emph{Subsequence} of a sequence $\{x_n\}_{n\in\mathbb N}$, we shall mean a sequence of the form $\{x_nk\}_{k\in \mathbb N}$, where each $n_k\in \mathbb N$ and $n_1<n_2<...$
\end{defbox}
\paragraph{•} Thus a subsequence $x_{n1},x_{x2},....$ of $x_1,x_2,...$ is obtained by "deleting" from $x_1,x_2,...$ all $x_n$'s except those such that $n=nk$\\
\paragraph{•} Subsequence are sometimes used to correct a sequence that behaves badly or to speed up convergence of another that converges slowly.


\subsection{Example}
\begin{enumerate}
    \item Prove that ${1\over n} \to 0$ as $n\to \infty$
   \item if $x_n \to 2$, prove that ${(2x_n + 1)\over x_n} \to {5\over 2}$ as $n\to \infty$
\end{enumerate}
\paragraph{Proof 1)}
Let $\varepsilon >$ 0. Use the Archimedean Principle\footnote{Definition An ordered field F has the Archimedean Property if, given any positive x and y in F there is an integer n $>$ 0 so that nx $>$ y.} to choose $N\in \mathbf{N}$ such that $N > {1\over \varepsilon}$. By taking the reciprocal of this inequality. We see that $n \geq N$ implies ${1\over n}\leq{1\over N}<\varepsilon$. \\
\underline{Since $1\over n$ are all positive, it follows that $|{{1\over n} < \varepsilon}|$ for all $n\geq N$}
\paragraph{Proof 2)}
Let $\varepsilon > 0$ \\Since $x_n \to 2$, apply Definition 2.1 to this $\varepsilon > 0$ to choose $N_1 \in \mathbf{N}$ such that $n\geq N_1$ implies $|x_n - 2|<\varepsilon$\\Next, apply Definition 2.1 with $\varepsilon = 1$ to choose $N_2$ such that $n\geq N_2$ implies $|x_n - 2| < 1$\\
By Fundamental Theorem of Absolute Values,we have $n\geq N_2$ implies $x_n >1$\\
Set $N = \max\{N_1,N-2\}$ and suppose that $n\geq N$. \\
Since $n\geq N_1$, we have $|2-x_n| = |x_n-2| < \varepsilon$\\
Since $n\geq N_2$, we have $0<{1\over (2x_n)}< {1\over 2} < 1$ It follows that
$$|{{{2x_n+1}\over{x_n}} - {5\over 2}}| = {{|2-x_n|}\over {2x_n}} <{{\varepsilon}\over {2x_n}} <\varepsilon$$
for all $x\geq N$

\subsection{Example}
\begin{enumerate}
    \item Prove the the sequnece $\{(-1)^n\}_{n\in \mathbb{N}}$ has no limits
  
\end{enumerate}
\paragraph{Proof}
Suppose that $(-1)^n \to a$ as $n\to \infty$ for some $a\in R$. Given $\varepsilon = 1$, there i s an $N\in \mathbb N$ such that $n \geq N$ implies $|(-1)^n -a| <\varepsilon$\\
For $n$ odd, this implies \ibx{$|1+a| = |-1-a| < 1$}\\
For $n$ even, this implies \ibx{$|1-a| < 1$}\\
$$ 2 = |1+1|\leq{|1-a|+|1+a|}<1 + 1= 2$$
that is, 2$<$ 2, a contradiction\footnote{Triangle Inequality}
\subsection{}






\chapter{Proof Technique}
\section{Limits}
\subsection{Limits Not Exist}
\begin{enumerate}
    \item Prove the the sequnece $\{(-1)^n\}_{n\in \mathbb{N}}$ has no limits
  
\end{enumerate}
\paragraph{Proof}
Suppose that $(-1)^n \to a$ as $n\to \infty$ for some $a\in R$. Given $\varepsilon = 1$, there i s an $N\in \mathbb N$ such that $n \geq N$ implies $|(-1)^n -a| <\varepsilon$\\
For $n$ odd, this implies \ibx{$|1+a| = |-1-a| < 1$}\\
For $n$ even, this implies \ibx{$|1-a| < 1$}\\
$$ 2 = |1+1|\leq{|1-a|+|1+a|}<1 + 1= 2$$
that is, 2$< $2, a contradiction\footnote{Triangle Inequality}

\part{newpart}

\tikzset{every picture/.style={line width=0.75pt}} %set default line width to 0.75pt        

\begin{tikzpicture}[x=0.75pt,y=0.75pt,yscale=-1,xscale=1]
%uncomment if require: \path (0,523); %set diagram left start at 0, and has height of 523
\node[




\end{tikzpicture}
\end{document}